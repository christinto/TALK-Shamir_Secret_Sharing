%
% Calculate period of CPU clock cycles.
%
\documentclass[preview,border=3mm]{article}
\usepackage{amsmath}
\usepackage{fancyhdr}
\usepackage{lipsum}

%\textheight=10in \textwidth=7in

\pagestyle{fancy}
\renewcommand{\sectionmark}[1]{\markright{#1}}
\fancyhf{}
\rhead{\fancyplain{}{}}
\lhead{\fancyplain{}{\rightmark }}
\cfoot{\fancyplain{}{\thepage}}

\begin{document}

% Custom equation counter. Currently does nothing different.
\setcounter{equation}{0}
\renewcommand{\theequation}{\arabic{equation}}

\section{Linear Polynomial}

%\lipsum
\noindent
A first degree polynomial is represented by the equation

\begin{align*}
    f(x) &= ax + b
\end{align*}

\noindent
Given our secret 42 and a random $a = 4$ we can represent a straight line

\begin{align}
    f(x) &= 4x + 42 \label{eq1}
\end{align}

\noindent
Suppose we have a single point on this line, say $P = (-17, -26)$, we want to
want to find an equation that describes all possible lines that include this
point.

\noindent
Given point $P$, we solve for $b$:


\begin{align}
    -26 &= -17a + b  \nonumber\\
    b   &= -26 + 17a \label{eq2}
\end{align}

\noindent
Now we may substitute $b$ from equation \ref{eq2} into equation \ref{eq1}.


\begin{align}
    f_a(x) &= ax -26 + 17a \nonumber \\
    f_a(x) &= a(x + 17) - 26 \label{eq3}
\end{align}

\noindent
Equation \ref{eq3} describes all lines passing through point $P = (-17, -26)$.\\


\section{Quadratic Polynomial}

\noindent
A second degree, or quadratic, polynomial is represented as

\begin{align}
    f(x) &= ax^2 + bx + c \label{eq4}
\end{align}

\noindent
Given a secret $S = 42$, we can choose random values for $a$ and $b$. Suppose
we choose $a = 7$ and $b = -3$. We may then write

\begin{align}
    f(x) &= 7x^2 - 3x + 42 \label{eq5}
\end{align}

\noindent
Three points must be given to describe a quadratic equation. Given only two
points, we want to find all quadratic curves that pass through the two given
points.\\

\noindent
Suppose we have points $P_1 = (2, 64)$ and $P_2 = (5, 202)$ discovered by choosing
two $x$ values randomly and solving equation \ref{eq5} for $f(x)$. We may find
the equation describing all curves passing through $P_1$ and $P_2$ by solving
the system of equations as we did in the case of the linear polynomial.\\

\noindent
Substitutine $P_1$ and $P_2$ into equation \ref{eq4} we can write

\begin{align}
    f(2) &= 4a + 2b + c \nonumber \\
    64   &= 4a + 2b + c \nonumber \\
    c    &= -4a - 2b + 64 \label{eq6}
\end{align}

\noindent
and

\begin{align}
    f(5) &= 25a + 5b + c \nonumber \\
    202  &= 25a + 5b + c \nonumber \\
    c    &= -25a - 5b + 202 \label{eq7}
\end{align}

\noindent
Since we know equations \ref{eq6} and \ref{eq7} are equal, we can solve for $b$
in terms of $a$ as follows:

\begin{align}
    -4a - 2b + 64 &= -25a - 5b + 202 \nonumber \\
    5b - 2b       &= 202 - 64 + 4a - 25a \nonumber \\
    3b            &= -21a + 138 \nonumber\\
    b             &= -7a + 46 \label{eq8}
\end{align}

\noindent
We can now substitute equation \ref{eq8} into equation \ref{eq6} to write:

\begin{align}
    c &= -25a -2(46 - 7a) + 64 \nonumber \\
      &= 14a - 4a - 92 +64 \nonumber \\
    c &= 10a - 28 \label{eq9}
\end{align}

\noindent
Now that we have solved for both $b$ and $c$ in terms of $a$, we may rewrite
equation \ref{eq4} as follows:

\begin{align}
    f_a(x) &= ax^2 + (-7a + 46)x + 10a - 28 \nonumber \\
           &= ax^2 + -7ax + 46x + 10a - 28 \nonumber \\
    f_a(x) &= a(x^2 - 7x + 10) + 46x -28 \label{eq10}
\end{align}

\noindent
Equation \ref{eq10} is the set of quadratic curves that pass through points
$P_1 = (2, 64)$ and $P_2 = (5, 202)$.


\section{Cubic Polynomial}

\noindent
A third degree, or cubic, polynomial is represented as:

\begin{align}
    f(x) &= ax^3 + bx^2 + cx + d \label{eq11}
\end{align}

\noindent
As before, our secret $S = 42$. We then select random values for $a$, $b$, and
$c$. Suppose we have $a = 1$, $b = 3$, and $c = -13$. We may write:

\begin{align}
    f(x) &= x^3 + 3x^2 - 13x + 42 \label{eq12}
\end{align}

\noindent
Four points are required to identify a specific cubic equation. Suppose we have
three points $P_1 = (-6, 12)$, $P_2 = (1, 33)$, $P_3 = (4, 102)$. We can find
the set of cubic polynomials that contains each of $P_1$, $P_2$, and $P_3$ by
solving the system of equations as before. We can write:

\begin{align}
    f(-6) &= a(-6)^3 + b(-6)^2 + c(-6) + d \nonumber \\
    12    &= -216a + 36b - 6c + d \nonumber \\
    d     &= 216a - 36b + 6c + 12 \label{eq13}
\end{align}

\begin{align}
    f(1) &= a + b + c + d \nonumber \\
    33   &= a + b + c + d \nonumber \\
    d    &= -a - b - c + 33\label{eq14}
\end{align}

\begin{align}
    f(4) &= 4^3a + 4^2b + 4c + d \nonumber \\
    102  &= 64a + 16b + 4c + d \nonumber \\
    d    &= -64a - 16b -4c + 102 \label{eq15}
\end{align}

\noindent
We may now represent $c$ in terms of $a$ and $b$ by simplifying the equality
of equations \ref{eq13} and \ref{eq14}:

\begin{align}
    216a - 36b + 6c + 12 &= -a -b -c + 33 \nonumber \\
    7c &= -217a + 35b + 21 \nonumber \\
    c &= -31a + 5b + 3 \label{eq16}
\end{align}

\noindent
And now repeat for equations \ref{eq14} and \ref{eq15}:

\begin{align}
    -a -b -c + 33 &= -64a -16b -4c + 102 \nonumber \\
    3c &= -63a -15b + 69 \nonumber \\
    c &= -21a -5b + 23 \label{eq17}
\end{align}

\noindent
Repeating once again to solve for $b$ by simplifying the equality represented
by the equations \ref{eq16} and \ref{eq17}:

\begin{align}
    -31a + 5b + 3 &= -21a -5b + 23 \nonumber \\
    10b &= 10a + 20 \nonumber \\
    b &= a + 2 \label{eq18}
\end{align}

\noindent
At this point, we may substitute equations \ref{eq18}, \ref{eq16}, and \ref{eq14}
into the general cubic polynomial formula \ref{eq11}. Terms with $b$ and $c$
must be replaced as needed in equations \ref{eq16} and \ref{eq14} so that we
have an equation in terms of $a$ and $x$ only. The algebra is simple but long
so here is the result without further ceremony:

\begin{align}
    f_a(x) &= a(x^3 + x^2 -26x + 24) + 2x^2 + 13x + 18 \label{eq19}
\end{align}


\section{Polynomials}
\noindent
This is the general form of a polynomial of degree $k$.
\begin{align*}
    f(x) &= a_0 + a_1x + a_2x^2 + a_3x^3 + \ldots + a_kx^k\\
\end{align*}

\noindent
Or, more generally\dots

\begin{align*}
    f(x) &= \sum\limits_{k=0}^n a_kx^k
\end{align*}

\end{document}
