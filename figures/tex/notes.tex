%
% Calculate period of CPU clock cycles.
%
\documentclass[preview,border=3mm]{article}
\usepackage{amsfonts}
\usepackage{amsmath}
\usepackage{fancyhdr}
\usepackage{lipsum}

\pagestyle{fancy}
\renewcommand{\sectionmark}[1]{\markright{#1}}
\fancyhf{}
\rhead{\fancyplain{}{}}
\lhead{\fancyplain{}{\rightmark }}
\cfoot{\fancyplain{}{\thepage}}

\begin{document}
%%%%%%%%%%%%%%%%%%%%%%
\section{Introduction}
%%%%%%%%%%%%%%%%%%%%%%
This document contains detailed notes related to the slides.


%%%%%%%%%%%%%%%%%%%%%
\section{Polynomials}
%%%%%%%%%%%%%%%%%%%%%
\noindent
This is the general form of a polynomial of degree $k$.
\begin{align*}
    f(x) &= a_0 + a_1x + a_2x^2 + a_3x^3 + \ldots + a_kx^k\\
\end{align*}

\noindent
Or, more generally\dots

\begin{align*}
    f(x) &= \sum\limits_{k=0}^n a_kx^k
\end{align*}


%%%%%%%%%%%%%%%%%%%%%%%
\section{Curve Fitting}
%%%%%%%%%%%%%%%%%%%%%%%
\noindent
A polynomial of degree $n$ may be uniquely defined given $n+1$ points on the
line. An infinite number of lines will intersect only $n$ points. These lines
may be found by solving the simultaneous equations for the points provided.

\noindent
These are the points used in the slides.

\subsection{Linear Polynomial}

%\lipsum
\noindent
A first degree polynomial is represented by the equation

\begin{align*}
    f(x) &= ax + b
\end{align*}

\noindent
Given our secret 42 and a random $a = 4$ we can represent a straight line

\begin{align}
    f(x) &= 4x + 42 \label{eq1}
\end{align}

\noindent
Suppose we have a single point on this line, say $P = (-17, -26)$, we want to
want to find an equation that describes all possible lines that include this
point.

\noindent
Given point $P$, we solve for $b$:


\begin{align}
    -26 &= -17a + b  \nonumber\\
    b   &= -26 + 17a \label{eq2}
\end{align}

\noindent
Now we may substitute $b$ from equation \ref{eq2} into equation \ref{eq1}.


\begin{align}
    f_a(x) &= ax -26 + 17a \nonumber \\
    f_a(x) &= a(x + 17) - 26 \label{eq3}
\end{align}

\noindent
Equation \ref{eq3} describes all lines passing through point $P = (-17, -26)$.\\


\subsection{Quadratic Polynomial}

\noindent
A second degree, or quadratic, polynomial is represented as

\begin{align}
    f(x) &= ax^2 + bx + c \label{eq4}
\end{align}

\noindent
Given a secret $S = 42$, we can choose random values for $a$ and $b$. Suppose
we choose $a = 7$ and $b = 3$. We may then write

\begin{align}
    f(x) &= 7x^2 + 3x + 42 \label{eq5}
\end{align}

\noindent
Three points must be given to describe a quadratic equation. Given only two
points, we want to find all quadratic curves that pass through the two given
points.\\

\noindent
Suppose we have points $P_1 = (2, 76)$ and $P_2 = (5, 232)$ discovered by choosing
two $x$ values randomly and solving equation \ref{eq5} for $f(x)$. We may find
the equation describing all curves passing through $P_1$ and $P_2$ by solving
the system of equations as we did in the case of the linear polynomial.\\

\noindent
Substitutine $P_1$ and $P_2$ into equation \ref{eq4} we can write

\begin{align}
    f(2) &= 4a + 2b + c \nonumber \\
    76   &= 4a + 2b + c \nonumber \\
    c    &= -4a - 2b + 76 \label{eq6}
\end{align}

\noindent
and

\begin{align}
    f(5) &= 25a + 5b + c \nonumber \\
    232  &= 25a + 5b + c \nonumber \\
    c    &= -25a - 5b + 232 \label{eq7}
\end{align}

\noindent
Since we know equations \ref{eq6} and \ref{eq7} are equal, we can solve for $b$
in terms of $a$ as follows:

\begin{align}
    -4a - 2b + 76 &= -25a - 5b + 232 \nonumber \\
    5b - 2b       &= 232 - 76 + 4a - 25a \nonumber \\
    3b            &= -21a + 156 \nonumber\\
    b             &= -7a + 52 \label{eq8}
\end{align}

\noindent
We can now substitute equation \ref{eq8} into equation \ref{eq6} to write:

\begin{align}
    c &= -4a - 2(-7a + 52) + 76 \nonumber \\
      &= -4a + 14a - 104 + 76 \nonumber \\
    c &= 10a - 28 \label{eq9}
\end{align}

\noindent
Now that we have solved for both $b$ and $c$ in terms of $a$, we may rewrite
equation \ref{eq4} as follows:

\begin{align}
    f_a(x) &= ax^2 + (-7a + 52)x + 10a - 28 \nonumber \\
           &= ax^2 + -7ax + 52x + 10a - 28 \nonumber \\
    f_a(x) &= a(x^2 - 7x + 10) + 52x - 28 \label{eq10}
\end{align}

\noindent
Equation \ref{eq10} is the set of quadratic curves that pass through points
$P_1 = (2, 76)$ and $P_2 = (5, 232)$.


\subsection{Cubic Polynomial}

\noindent
A third degree, or cubic, polynomial is represented as:

\begin{align}
    f(x) &= ax^3 + bx^2 + cx + d \label{eq11}
\end{align}

\noindent
As before, our secret $S = 42$. We then select random values for $a$, $b$, and
$c$. Suppose we have $a = 1$, $b = 3$, and $c = 13$. We may write:

\begin{align}
    f(x) &= x^3 + 3x^2 + 13x + 42 \label{eq12}
\end{align}

\noindent
Four points are required to identify a specific cubic equation. Suppose we have
three points $P_1 = (-6, -144)$, $P_2 = (1, 59)$, $P_3 = (4, 206)$. We can find
the set of cubic polynomials that contains each of $P_1$, $P_2$, and $P_3$ by
solving the system of equations as before. We can write:

\begin{align}
    f(-6) &= a(-6)^3 + b(-6)^2 + c(-6) + d \nonumber \\
    -144  &= -216a + 36b - 6c + d \nonumber \\
    d     &= 216a - 36b + 6c - 144 \label{eq13}
\end{align}

\begin{align}
    f(1) &= a + b + c + d \nonumber \\
    59   &= a + b + c + d \nonumber \\
    d    &= -a - b - c + 59\label{eq14}
\end{align}

\begin{align}
    f(4) &= 4^3a + 4^2b + 4c + d \nonumber \\
    206  &= 64a + 16b + 4c + d \nonumber \\
    d    &= -64a - 16b - 4c + 206 \label{eq15}
\end{align}

\noindent
We may now represent $c$ in terms of $a$ and $b$ by simplifying the equality
of equations \ref{eq13} and \ref{eq14}:

\begin{align}
    216a - 36b + 6c - 144 &= -a -b -c + 59 \nonumber \\
    7c &= -217a + 35b + 203 \nonumber \\
    c &= -31a + 5b + 29 \label{eq16}
\end{align}

\noindent
And now repeat for equations \ref{eq14} and \ref{eq15}:

\begin{align}
    -a - b - c + 59 &= -64a - 16b - 4c + 206 \nonumber \\
    3c &= -63a - 15b + 147 \nonumber \\
    c &= -21a - 5b + 49 \label{eq17}
\end{align}

\noindent
Repeating once again to solve for $b$ by simplifying the equality represented
by the equations \ref{eq16} and \ref{eq17}:

\begin{align}
    -31a + 5b + 29 &= -21a - 5b + 49 \nonumber \\
    10b &= 10a + 20 \nonumber \\
    b &= a + 2 \label{eq18}
\end{align}

\noindent
At this point, we may substitute equations \ref{eq18}, \ref{eq16}, and \ref{eq14}
into the general cubic polynomial formula \ref{eq11}. Terms with $b$ and $c$
must be replaced as needed in equations \ref{eq16} and \ref{eq14} so that we
have an equation in terms of $a$ and $x$ only. The algebra is simple but long
so here is the result without further ceremony:

\begin{align}
    f_a(x) &= a(x^3 + x^2 - 26x + 24) + 2x^2 + 39x + 18 \label{eq19}
\end{align}


%%%%%%%%%%%%%%%%%%%%%%%%%%%%%%%
\section{Shamir Secret Sharing}
%%%%%%%%%%%%%%%%%%%%%%%%%%%%%%%

\subsection{Mathematical Definition}

Divide secret $S$ into $n$ parts $S_1,\ldots,S_n$ such that

\begin{itemize}
    \item Knowledge of any $k$ or more $S_i$ makes secret $S$ computable.
    \item Knowledge of any $k-1$ or less $S_i$ leaves secret $S$ completely undetermined.
\end{itemize}

\noindent
This is called a $(k,n)$ threshold scheme.\\

\noindent
The essential idea is that $k$ points are required to define a polynomial of
degree $k-1$.\\

\noindent
For our purposes we can define a polynomial as follows:

\begin{align*}
    f(x) &= a_0 + a_1x + a_2x^2 + \ldots + a_{k-1}x^{k-1} \\
    f(x0 &= \sum_{i=0}^{k-1} a_ix^i
\end{align*}

\noindent
Suppose we want to use a $(k,n)$ threshold scheme to share our secret $S$, an
element in a finite field $\mathbb{F}$ of size $P$ where
$0 < k \leq n < P; S < P$ and $P$ is a prime number.

\begin{itemize}
    \item Choose at random $k-1$ positive integers $a_1, \ldots a_{k-1}$
        with $0 < a_i < P; a_i \in \mathbb{N}$ and let $a_0 = S; S < P$.
    \item Build the polynomial
        $f(x) = a_0 + a_1x + a_2x^2 + \ldots + a_{k-1}x^{k-1} \mod P$
    \item Construct any $n$ points, \emph{for instance}, set $i = 1, \ldots , n$
        to retrieve $(i, f(i))$.
\end{itemize}

\noindent
Every participant is given a point, a point, e.g.\ in integer input to the
polynomial and the corresponding integer output. \\

\noindent
Given any subset $k$ of these pairs, we can find the coefficients of the
polynomial. The secret is the constant term $a_0$.


\subsection{Examples}

\subsubsection{The Probem}
In this example we will omit the requirement that $\mod P$ is applied to
the polynomial. \\

\noindent
Suppose Eve has managed to obtain a share, say $P = (16, 106)$ which is ak
point on the curve defined by \ref{eq1}. Eve knows $f(x) = ax + b$. \\

\noindent
She can solve for $b$ in terms of $a$.

\begin{align}
    f(x) &= ax + b \nonumber \\
    P \rightarrow f(16) = 106 &= a(16) + b \nonumber \\
    b &= -16a + 106 \label{eq20}
\end{align}

\noindent
Recall the requirement that $a,b \in \mathbb{N}$.\\

\noindent
Eve may then substitute values for unknown $a$ in equation \ref{eq20}.

\begin{align*}
    a = 0 \rightarrow b &= -16(0) + 106 = 106 \\
    a = 1 \rightarrow b &= -16(1) + 106 = 90 \\
    a = 2 \rightarrow b &= -16(2) + 106 = 74 \\
    a = 3 \rightarrow b &= -16(3) + 106 = 58 \\
    a = 4 \rightarrow b &= -16(4) + 106 = 42 \\
    a = 5 \rightarrow b &= -16(5) + 106 = 26 \\
    a = 6 \rightarrow b &= -16(6) + 106 = 10 \\
    a = 7 \rightarrow b &= -16(7) + 106 = -6
\end{align*}

\noindent
Since the requirement is that $a,b \in \mathbb{N}$, a cannot be negative.
Therefore, Eve can conclude

\begin{align*}
    a &\in [0, 1, 2, 3, 4, 5, 6] \\
    b &\in [106, 90, 74, 58, 42, 26, 10]
\end{align*}


\subsubsection{The Solution}
The solution is to require that $S$ is an element in a finite field $\mathbb{F}$
of size $P$ where $S < P$ and $P$ is prime.\\

\noindent
Since $S = 42$, choosing $P = 43$ satisfies the requirement that $S < P$ and
$P$ is prime.

\begin{align*}
    f(x)  &= 4x + 42 \mod 43 \\
    f(16) &= 4(16) + 42 \mod 43 \\
          &= 106 \mod 43 \\
    f(16) &= 20
\end{align*}

\noindent
Recall from the definition of a modulus that

$$a \mod P = a - Pm | 0 \leq a - pm \leq P$$

\noindent
In other words, $m$ is a multiplier. \\

\noindent
Eve knows $P = (16, 20)$, $f(x) = ax + b \mod P$, and modulus $P = 43$. \\

\noindent
So she can substitute and write

\begin{align}
    f(x) &= ax + b \mod P \nonumber \\
         &= ax + b - pm \nonumber \\
      20 &= 16a + b - 43m \nonumber \\
      b  &= -16a + 20 + 43m \label{eq21}
\end{align}

\noindent
As before, Eve can then substitute values for $a$ into equation \ref{eq21}.

\begin{align}
    a = 0 \rightarrow b = -16(0) + 20 + 43m &= 20 + 43m \nonumber \\
    a = 1 \rightarrow b = -16(1) + 20 + 43m &= 4 + 43m \nonumber \\
    a = 2 \rightarrow b = -16(2) + 20 + 43m &= -12 + 43m \nonumber \\
    a = 3 \rightarrow b = -16(3) + 20 + 43m &= -28 + 43m \nonumber \\
    a = 4 \rightarrow b = -16(4) + 20 + 43m &= -44 + 43m \label{eq22} \\
    a = 5 \rightarrow b = -16(5) + 20 + 43m &= -60 + 43m \nonumber \\
    a = 6 \rightarrow b = -16(6) + 20 + 43m &= -76 + 43m \nonumber \\
    a = 7 \rightarrow b = -16(7) + 20 + 43m &= -92 + 43m \nonumber
\end{align}

\noindent
This time since $m$ is unknown, Eve can gain no information about the secret.
But since we know $a = 4$ is the correct value, we can see that equation
\ref{eq22} is the correct result. In other words, $m = 2$. \\

\noindent
The secret $S$ is equally likely to be any element of the finite field
$\mathbb{F}$. No information may be learned about $S$ unless $S_1 \ldots S_k$
are known.

\end{document}
